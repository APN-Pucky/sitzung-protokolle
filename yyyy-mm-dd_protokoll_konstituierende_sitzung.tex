\documentclass[sitzung=fsv-konstituierend]{fsphys-protokoll}

%%% Datum, Uhrzeit, Teilnehmer, Ort
\renewcommand*{\protokolldatum}{yyyy-mm-dd}
\renewcommand*{\protokollbeginn}{hh:mm~Uhr}
\renewcommand*{\protokollende}{hh:mm~Uhr}
\renewcommand*{\protokollant}{Name des Protokollanten}
\renewcommand{\protokollort}{StudiO, Wilhelm-Klemm-Str.~9, 48149~Münster}
% Listen unbedingt durch Kommata trennen, sonst wird nicht richtig gezählt.
% Anwesende Mitglieder der FSV
\renewcommand{\protokollanwesend}{Namen}
% Weitere Anwesende auf der Sitzung
\renewcommand{\protokollweitere}{Namen}

\begin{document}
	
\section{Begrüßung und Wahl einer Versammlungsleitung und eines Protokollanten}
\begin{itemize}
	\item XXX begrüßt alle Anwesenden.
	\item YYY wird mit 00J:00N:00E\footnote{Hinweis zur Konvention: Ergebnisse von Abstimmungen werden im in der Reihenfolge (Ja-Stimmen:Nein-Stimmen:Enthaltungen) angegeben.} zur Versammlungsleitung gewählt.
	\item ZZZ wird mit 00J:00N:00E zum Protokollanten gewählt.
\end{itemize}

\section{Feststellung der Beschlussfähigkeit}
Für die Beschlussfähigkeit müssen mindestens \SI{50}{\percent} der wahlberechtigten Mitglieder anwesend sein.

Die Beschlussfähigkeit wird mit XX anwesenden von insgesamt 15~Mitgliedern der FSV festgestellt.

\section{Feststellung der Tagesordnung}
Festgestellt wie vorgestellt.

\section{Wahl des neuen Fachschaftsrats}
\begin{tabular}{| r @{ } l | p{5cm} | p{3cm} |}
\hline
& "Amt" (kann entfallen)
& \centering\let\newline\\\arraybackslash Vor- und Nachname;\newline Matrikel-Nr.
& \centering\let\newline\\\arraybackslash Abstimmung\newline(Ja:Nein:Enth.)
\\ \hline\hline
1.  & Vorsitzende/r & &
\\ \hline
2.  & Stellv. Vorsitzende/r & &
\\ \hline
3.  & Finanzreferent/in & &
\\ \hline
4.  & Stellv. Finanzreferent/in & &
\\ \hline
5.  & Erstibeauftragte/r & &
\\ \hline
6.  & Beauftragte/r O-Woche & &
\\ \hline
7.  & Stellv. Beauftragte/r O-Woche & &
\\ \hline
8.  & Beauftragte/r Ersti-Fahrt & &
\\ \hline
9.  & Stellv. Beauftragte/r Ersti-Fahrt & &
\\ \hline
10. & Beauftragte/r Ersti-Fibel & &
\\ \hline
11. & Stellv. Beauftragte/r Ersti-Fibel & &
\\ \hline
12. & Evaluationsbeauftragte/r & &
\\ \hline
13. & Stellv. Evaluationsbeauftragte/r & &
\\ \hline
14. & Stellv. Evaluationsbeauftragte/r & &
\\ \hline
15. & Beauftragte/r Sommerfest & &
\\ \hline
16. & Stellv. Beauftragte/r Sommerfest & &
\\ \hline
17. & Stellv. Beauftragte/r Sommerfest & &
\\ \hline
18. & Eventmanager/in & &
\\ \hline
19. & Stellv. Eventmanager/in & &
\\ \hline
20. & E-Mail-Beauftragte/r & &
\\ \hline
21. & Beauftragte/r BaMa-Tag & &
\\ \hline
22. & Stellv. Beauftragte/r BaMa-Tag & &
\\ \hline
23. & Beauftragte/r Buchmarkt & &
\\ \hline
24. & Stellv. Beauftragte/r Buchmarkt & &
\\ \hline
25. & Beauftragte/r ZaPF/Vernetzung & &
\\ \hline
26. & Beauftragte/r E-Learning & &
\\ \hline
27. & Beauftragte/r Protokoll-/Klausurverleih & &
\\ \hline
28. & Beauftragte/r Protokoll-/Klausurverleih & &
\\ \hline
29. & Pizzabeauftragte/r & &
\\ \hline
\end{tabular}

\bigskip
Das Protokoll wurde genehmigt auf der Sitzung am \protokolldatum.

\bigskip
\bigskip
Unterschrift der Protokollführerin/des Protokollführers

\end{document}
