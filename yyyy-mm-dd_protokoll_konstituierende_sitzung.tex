\documentclass[sitzung=fsv-konstituierend]{fsphys-protokoll}

%%% Datum, Uhrzeit, Teilnehmer, Ort
\renewcommand*{\protokolldatum}{yyyy-mm-dd}
\renewcommand*{\protokollbeginn}{hh:mm~Uhr}
\renewcommand*{\protokollende}{hh:mm~Uhr}
\renewcommand*{\protokollant}{Name des Protokollanten}
\renewcommand*{\protokollort}{StudiO, Wilhelm-Klemm-Str.~9, 48149~Münster}
% Listen unbedingt durch Kommata trennen, sonst wird nicht richtig gezählt.
% Anwesende Mitglieder der FSV
\renewcommand*{\protokollanwesend}{Namen}
% Weitere Anwesende auf der Sitzung
\renewcommand*{\protokollweitere}{Namen}

\begin{document}
	
\section{Begrüßung und Wahl einer Versammlungsleitung und eines Protokollanten}
\begin{itemize}
	\item XXX begrüßt alle Anwesenden.
	\item YYY wird mit 00J:00N:00E\footnote{Hinweis zur Konvention: Ergebnisse von Abstimmungen werden in der Reihenfolge (Ja-Stimmen:Nein-Stimmen:Enthaltungen) angegeben.} zur Versammlungsleitung gewählt.
	\item ZZZ wird mit 00J:00N:00E zum Protokollanten/zur Protokollantin gewählt.
\end{itemize}

\section{Feststellung der Beschlussfähigkeit}
Für die Beschlussfähigkeit müssen mindestens \SI{50}{\percent} der wahlberechtigten Mitglieder anwesend sein.

Die Beschlussfähigkeit wird mit {\protokollanzahlanwesend} anwesenden von insgesamt 15~Mitgliedern der Fachschaftsvertretung festgestellt.

\section{Feststellung der Tagesordnung}
Festgestellt wie vorgestellt.

\section{Wahl des neuen Fachschaftsrats}
\newcounter{fsr}
\newcommand*{\fsrnum}{\stepcounter{fsr}\arabic{fsr}.}
\begin{longtable}{| r @{ } l | l | c |}
	\hline
	& "Amt" (kann entfallen)
	& \parbox[t]{5cm}{\centering Vor- und Nachname;\\ Matrikel-Nr.}
	& \parbox[t]{2.8cm}{\centering Abstimmung\\(Ja:Nein:Enth.)}
	\\ \hline\hline
	\endhead
	\fsrnum & Vorsitzende/r
		& &
	\\ \hline
	\fsrnum & Stellv.\ Vorsitzende/r
		& &
	\\ \hline
	\fsrnum & Finanzreferent/in
		& &
	\\ \hline
	\fsrnum & Stellv.\ Finanzreferent/in
		& &
	\\ \hline
	\fsrnum & Evaluationsbeauftragte/r
		& &
	\\ \hline
	\fsrnum & Stellv.\ Evaluationsbeauftragte/r
		& &
	\\ \hline
	\fsrnum & Stellv.\ Evaluationsbeauftragte/r
		& &
	\\ \hline
	\fsrnum & Ersti-Beauftragte/r
		& &
	\\ \hline
	\fsrnum & Stellv.\ Ersti-Beauftragte/r
		& &
	\\ \hline
	\fsrnum & Beauftragte/r O-Woche
		& &
	\\ \hline
	\fsrnum & Stellv.\ Beauftragte/r O-Woche
		& &
	\\ \hline
	\fsrnum & Beauftragte/r Ersti-Fahrt
		& &
	\\ \hline
	\fsrnum & Stellv.\ Beauftragte/r Ersti-Fahrt
		& &
	\\ \hline
	\fsrnum & Beauftragte/r Ersti-Fibel
		& &
	\\ \hline
	\fsrnum & Stellv.\ Beauftragte/r Ersti-Fibel
		& &
	\\ \hline
	\fsrnum & Beauftragte/r Sommerfest
		& &
	\\ \hline
	\fsrnum & Stellv.\ Beauftragte/r Sommerfest
		& &
	\\ \hline
	\fsrnum & Stellv.\ Beauftragte/r Sommerfest
		& &
	\\ \hline
	\fsrnum & Eventmanager/in
		& &
	\\ \hline
	\fsrnum & Stellv.\ Eventmanager/in
		& &
	\\ \hline
	\fsrnum & Beauftragte/r BaMa-Tag
		& &
	\\ \hline
	\fsrnum & Stellv.\ Beauftragte/r BaMa-Tag
		& &
	\\ \hline
	\fsrnum & Beauftragte/r Buchmarkt
		& &
	\\ \hline
	\fsrnum & Stellv.\ Beauftragte/r Buchmarkt
		& &
	\\ \hline
	\fsrnum & Beauftragte/r Spieleabend
		& &
	\\ \hline
	\fsrnum & Stellv.\ Beauftragte/r Spieleabend
		& &
	\\ \hline
	\fsrnum & Beauftragte/r Protokoll-/Klausurverleih
		& &
	\\ \hline
	\fsrnum & Stellv.\ Beauft.\ Protokoll-/Klausurverleih
		& &
	\\ \hline
	\fsrnum & Lehrpreis-Komitee
		& &
	\\ \hline
	\fsrnum & Lehrpreis-Komitee
		& &
	\\ \hline
	\fsrnum & Lehrpreis-Komitee
		& &
	\\ \hline
	\fsrnum & Beauftragte/r Öffentlichkeitsarbeit
		& &
	\\ \hline
	\fsrnum & Stellv.\ Beauft.\ Öffentlichkeitsarbeit
		& &
	\\ \hline
	\fsrnum & E-Mail-Beauftragte/r
		& &
	\\ \hline
	\fsrnum & Stellv.\ E-Mail-Beauftragte/r
		& &
	\\ \hline
	\fsrnum & Imperia-Beauftragte/r
		& &
	\\ \hline
	\fsrnum & Stellv.\ Imperia-Beauftragte/r
		& &
	\\ \hline
	\fsrnum & PC-Administrator/in
		& &
	\\ \hline
	\fsrnum & PC-Administrator/in
		& &
	\\ \hline
	\fsrnum & Beauftragte/r Hochschultag/Schüler
		& &
	\\ \hline
	\fsrnum & Stellv.\ Beauft.\ Hochschultag/Schüler
		& &
	\\ \hline
\end{longtable}

\clearpage

\appendix

\section{Mitgliederliste des Fachschaftsrats}
\setcounter{fsr}{0}
\begin{longtable}{| r @{ } l | >{\raggedright\arraybackslash}m{10cm} |}
	\hline
	& Vor- und Nachname & "Ämter"
	\\ \hline\hline
	\endhead
	\fsrnum & 
	&
	\\ \hline
	\fsrnum & 
	&
	\\ \hline
	\fsrnum & 
	&
	\\ \hline
	\fsrnum & 
	&
	\\ \hline
	\fsrnum & 
	&
	\\ \hline
	\fsrnum & 
	&
	\\ \hline
	\fsrnum & 
	&
	\\ \hline
	\fsrnum & 
	&
	\\ \hline
	\fsrnum & 
	&
	\\ \hline
	\fsrnum & 
	&
	\\ \hline
	\fsrnum & 
	&
	\\ \hline
	\fsrnum & 
	&
	\\ \hline
	\fsrnum & 
	&
	\\ \hline
	\fsrnum & 
	&
	\\ \hline
	\fsrnum & 
	&
	\\ \hline
	\fsrnum & 
	&
	\\ \hline
	\fsrnum & 
	&
	\\ \hline
	\fsrnum & 
	&
	\\ \hline
	\fsrnum & 
	&
	\\ \hline
	\fsrnum & 
	&
	\\ \hline
\end{longtable}

\bigskip
\bigskip
Das Protokoll wurde genehmigt auf der Sitzung am \protokollformatteddate.

\vspace{2cm}
Unterschrift des Protokollführers/der Protokollführerin

\end{document}
