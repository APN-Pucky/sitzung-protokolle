% Entkommentieren für nicht oeffentliches Protokoll. 
% Die finale Fassung auf der FS-Platte sollte wieder kommentiert, also nicht privat (oeffentlich), sein. 
%\newif\ifprotokollprivate\protokollprivatetrue

\documentclass[sitzung=fsv,entwurf]{fsphys-protokoll}

%%% Datum, Uhrzeit, Teilnehmer
\renewcommand*{\protokolldatum}{yyyy-mm-dd}
\renewcommand*{\protokollbeginn}{hh:mm~Uhr}
\renewcommand*{\protokollende}{hh:mm~Uhr}
\renewcommand*{\protokollant}{Name des Protokollanten}
% Listen unbedingt durch Kommata trennen, sonst wird nicht richtig gezählt.
% Anwesende Mitglieder der FSV
\renewcommand*{\protokollanwesend}{Namen}
% Weitere Anwesende auf der Sitzung
\renewcommand*{\protokollweitere}{Namen}

\begin{document}

\section{Begrüßung und Wahl eines Protokollanten}
\begin{itemize}
	\item XXX begrüßt alle Anwesenden.
	\item YYY wird mit 00J:00N:00E zum Protokollanten/zur Protokollantin gewählt.
\end{itemize}

\section{Feststellung der Beschlussfähigkeit}
Für die Beschlussfähigkeit müssen mindestens \SI{50}{\percent} der wahlberechtigten Mitglieder anwesend sein.

Die Beschlussfähigkeit wird mit {\protokollanzahlanwesend} anwesenden von insgesamt 15~Mitgliedern der Fachschaftsvertretung festgestellt.

% Hier weitere TOPs einfügen…

\end{document}
