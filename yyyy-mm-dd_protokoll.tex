\documentclass[sitzung=fsr]{fsphys-protokoll}

%%% Datum, Uhrzeit, Teilnehmer
\renewcommand*{\protokolldatum}{YYYY-MM-DD}
\renewcommand*{\protokollbeginn}{hh:mm~Uhr}
\renewcommand*{\protokollende}{hh:mm~Uhr}
\renewcommand*{\protokollant}{Name des Protokollanten}
% Listen unbedingt durch Kommata trennen, sonst wird nicht richtig gezählt.
% Anwesende FSR-Mitglieder
\renewcommand*{\protokollanwesend}{Namen}
% Entschuldigte
\renewcommand*{\protokollfehlend}{Namen}
% Anwesende Gäste
\renewcommand*{\protokollweitere}{Namen}

\begin{document}

\section{Genehmigung des letzten Protokolls und des Putzdiensts}
\begin{itemize}
	\item Die Beschlussfähigkeit wird gemäß §~3 Abs.~2 der Geschäftsordnung des Fachschaftsrats Physik mit {\protokollanzahlanwesend} anwesenden von X Mitgliedern des Fachschaftsrats festgestellt.
	\item 
\end{itemize}

\section{Berichte}
\begin{itemize}
	\item 
\end{itemize}

\section{E-Mail-Anfragen}
\begin{itemize}
	\item 
\end{itemize}

\section{Kommissionen und Fachbereichsrat~(FBR)/Senat}
\begin{itemize}
	\item 
\end{itemize}

\section{Fachschaftenkonferenz~(FK)}
\begin{itemize}
	\item 
	\item Ansonsten wird auf das \href{https://www.asta.ms/hochschulpolitik/hochschulpolitik/fachschaften/fk-protokolle}{FK-Protokoll} verwiesen.
\end{itemize}

\section{Verschiedenes}
\begin{itemize}
	\item 
\end{itemize}

\end{document}
